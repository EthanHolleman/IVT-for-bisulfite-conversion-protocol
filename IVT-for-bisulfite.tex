\documentclass[12pt]{article}
\usepackage[utf8]{inputenc}	% Para caracteres en español
\usepackage{amsmath,amsthm,amsfonts,amssymb,amscd}
\usepackage{multirow,booktabs}
\usepackage[table]{xcolor}
\usepackage{fullpage}
\usepackage{lastpage}
\usepackage{enumitem}
\usepackage{fancyhdr}
\usepackage{mathrsfs}
\usepackage{wrapfig}
\usepackage{graphicx}
\usepackage{caption}
\usepackage{subcaption}
\usepackage{setspace}
\usepackage{calc}
\usepackage{multicol}
\usepackage{cancel}
\usepackage[T1]{fontenc} 
%\usepackage[retainorgcmds]{IEEEtrantools}
\usepackage[margin=3cm]{geometry}
\usepackage{amsmath}
\newlength{\tabcont}
\setlength{\parindent}{0.0in}
\setlength{\parskip}{0.05in}
\usepackage{empheq}
\usepackage{framed}
\usepackage[most]{tcolorbox}
\usepackage{xcolor}
\colorlet{shadecolor}{orange!15}
\parindent 0in
\parskip 12pt
\usepackage[T1]{fontenc}

 \renewcommand{\familydefault}{\sfdefault}
\geometry{margin=1in, headsep=0.25in}
\theoremstyle{definition}


\usepackage{graphicx}
\usepackage{hyperref}
\hypersetup{
	colorlinks=true,
	linkcolor=blue,
	filecolor=magenta,      
	urlcolor=cyan,
}

\begin{document}
\title{
	\textbf{\emph{In vitro} transcription for bisulfite conversion}\\
	\large Adapted from \emph{Malig et al, 2020}}

\author{Ethan Holleman}
\maketitle

\section*{Background}

This protocol describes a general method of transcribing plasmid or linear DNA molecules \emph{in vitro} to observe R-loop formation through agarose gel electrophorsis and prepare samples
for single molecule R-loop footprinting by bisulfite conversion. Specific volumes for reagents are not used unless they remain
constant and instead a spreadsheet that will help you determine
these values depending on your specific experiment is provided in this documents repository in the resources directory.

\section*{Protocol}

\subsection*{IVT}

\begin{enumerate}

\item Determine reaction volume and DNA mass per plasmid. This will be dependent on
what intended use for the IVT products is, and the number of reactions per plasmid. Generally, a complete IVT to demonstrate R-loop formation (or lack thereof) via band shift on an agarose gel will have three samples per plasmid; an untranscribed control, a transcribed RnaseH treated and a transcribed untreated. A minimum of 200 ng per sample should be used in order to visualize on an agarose gel.

\item After determining reagent volumes for each sample, assemble reagents for each sample in the following order; npH$_{2}$0, 10X reaction buffer, NTPs, template DNA. This will serve as a master mix for all reactions utilizing a particular template DNA.\footnote{It can be helpful when dealing with many samples to prepare reactions in a PCR grid. This way the main mix can be created in one well and individual reactions can be split from the main mix into spacial related wells using a multichannel pipette.}

\item Split each sample into individual reactions (typically 2 to 3) of equal volume.

\item Add the appropriate polymerase for your template DNA (usually T7 or T3). 2 $\mu$l is suitable for up to 1 $\mu$g of DNA.\footnote{If larger masses of DNA are used, NEB's \href{https://www.neb.com/protocols/2015/03/09/protocol-for-standard-rna-synthesis}{Protocol for Standard RNA Synthesis} can be scaled up and used as a guide.}

\item Incubate all samples at 37$^{\circ}$ C for 20 minutes in a thermocycler.\footnote{Make sure an aliquot of EDTA is ready after you start the incubation.}

\item Immediately remove samples and place on ice. Add EDTA to all samples.\footnote{Final EDTA concentration should be 5x the concentration of Mg in the transcription buffer. NEB 10x transcription buffer contains 6 mM MgCl$_{2}$.}

\item Thoroughly mix samples and place on ice for 5 minutes.

\subsection*{Phenol / EtOH precipitation}

\item Increase sample volume to 200 $\mu$l with TE or 10 mM Tris HCl.

\item Spin down phase lock gel tube at 12,000 x g for 30 seconds. Add 200 $\mu$l Phenol:Chloroform:Isoamyl Alcohol (25:24:1, v/v) (1 volume) to the tube.

\item Add sample to phase lock tube and vortex until contents are thoroughly mixed. 

\item Centrifuge at 12,000 x g for 5 minutes to separate phases.

\item Carefully pipet off nucleic-acid-containing aqueous phase (upper layer) to a fresh tube.

\item Preform standard EtOH precipitation (\href{https://openwetware.org/wiki/Ethanol_precipitation_of_nucleic_acids}{OpenWetWare} has a useful guide with images). Add 0.1 volumes 3M sodium acetate pH 5.2, 2.2 volumes 100\% EtOH and 2 $\mu$l molecular grade glycogen to each sample. Freeze samples according to your needs. Generally, the lower the concentration of
DNA the greater benefit of longer freezing times. If maximum yield is required freeze overnight at -20$^{\circ}$C, if time is a
greater constraint freeze at -80$^{\circ}$ for 10 minutes to 1 hour.

\item Spin at full speed in microcentrifuge at 4$^{\circ}$ for 30 minutes. 

\item Carefully vacuum supernatant away. Wash twice with 200 $\mu$l 70\% EtOH. It can be helpful to spin samples briefly between washes to secure the pellet as it can sometimes be dislodged during the wash. Carefully vacuum away the 70\% EtOH
between each wash.

\item Air dry pellet for 20-30 minutes.

\item Re-suspend pellet in desired volume of 10 mM Tris HCl. This will be dependent on downstream use of the sample.

\subsection*{Visualize via agarose gel}

\item Prepare a 0.8\% 1x TBE agarose gel. Cover in 1x TBE. Do not add ethidium bromide to the gel or running buffer. 

\item Aliquot 200-600 ng of each sample into a PCR tube. Increase to suitable volume (usually between 10 and 20 $\mu$l depending on well size) and add appropriate amount of loading dye. 

\item Run gel at max 60V for minimum time of 2 hours.

\item Post stain gel for 1 hour with ethidium bromide at a concentration of 1 $\mu$l ethidium bromide per 100 ml running buffer. 

\item De-stain the gel for 10 minutes in np water.

\item Image the gel and record results.

\subsection*{Bisulfite conversion}

\item Follow protocol provided with Zymogen \href{https://www.zymoresearch.com/collections/ez-dna-methylation-lightning-kits/products/ez-dna-methylation-lightning-kit}{EZ DNA Methylation-Lightning Kit}(Cat \# D5030-E) except for the incubation step. It is critical that incubation with bisulfite be done under non-denaturing conditions in order for results to be informative to R-loop formation. Accordingly, samples should be incubated for 2 hours at 37$^{\circ}$C with rotation. 




\end{enumerate}


\end{document}

